\chapter[Introdução]{Introdução}\label{cap1}
		
Atualmente, a quantidade de dados disponíveis, na Internet, é muito grande, e vem aumentando rapidamente. Devido a este crescimento, torna-se fundamental a aplicação de técnicas para melhoria no tratamento e organização destes dados, atuando principalmente na sua seleção, processamento, recuperação e disseminação.
Segundo Almeida et al. \cite{UVG}, técnicas de tratamento e organização de dados podem ser classificadas de diversas formas, por exemplo, a partir de seus termos, em glossários ou dicionários, por classificação ou categorias, através de taxonomias, ou a partir de conceitos e seus relacionamentos, utilizando ontologias, tesauros ou redes semânticas.\cite{TSW}
Ontologia é uma técnica de organização de informações que vem recebendo especial atenção nos últimos anos, principalmente no que diz respeito à representação formal de conhecimento. Geralmente criadas por especialistas, tendo sua estrutura baseada na descrição de conceitos e dos relacionamentos semânticos entre eles, as ontologias geram uma especificação formal e explícita de uma conceitualização compartilhada.

\section{Apresentação do Documento do Projeto}

Este documento é resultado da disciplina Web Semântica do Curso de Engenharia de Software da Universidade de Brasília, campus Gama. E está organizado nos seguintes capítulos: Capitulo 1 – Introdução - contento a apresentação do problema, contexto e objetivos do projeto -  Capitulo 2 – Proposta - contento os a descrição da ontologias usadas, usuários e plataforma - Capitulo 3 – Conceitos abordados em sala.

\section{Descrição do Problema}
Nos dias atuais é comum ouvir discussões em relação à vida moderna, sobre as tendencias do mercado, sobre a Internet, de umaforma geral, discutese sobre a grande era digital. hoje, podese perceber que a forma de se relacionar mudou. Segundo \cite{FA}, a inflência do mundo globalizado, mudou os costumes familiares e colaboraram para que houvesse uma grade mudança na forma de comunicação da sociedade.

Olhando para o ambiênte acadêmico, identifica-se a mesma mudança. Há a tendência de que os alunos não busquem mais a resposta na pessoa do professor. Com isso novas metodologias de ensino estão sendo desenvolvidas, como por exemplo o CBL - Challenge Based Learning - que consiste em desafios, ou seja, não há uma aula como de costume mais há um desafio que deve ser solucionado e o grupo deve encontrar a solução\cite{CBL}.

Uma das ferramentas utilizadas em ambientes acadêmicos é o moodle. O moodle tem como objetivo criar redes de relacionamento virtual para o aprendizado. Existem, hoje, algumas ontologias que analisam a rede acadêmica criada pelo moodle. Porém, poucas dão recusos de buscas de alunos por um determinado perfil. Por exemplo, caso um docente da instituição queira desenvolver um projeto e esteja precisando de um determinado perfil de aluno, um aluno que tenha feito as disciplinas de Cálculo Numérico e que tenha interesse em processamento de imagens, ele teria que sair buscando na instituição sala por sala. A ideia principal desse trabalho é solucionar esse problema através da plataforma moodle.


\section{Contexto}

Segundo Breitman \cite{BK}, no inicio da internet as paginas de um site eram bastante simples e eram uma maneira fácil de compartilhar informações. Com o passar do tempo, a internet começou popularidade e com isso veio sua evolução. A web ( palavra de origem inglesa cujo a melhor tradução é Teia ou Rede ) ainda hoje continua a crescer e a se tornar cada vez mais presente na vida das pessoas.  Em 2012 a empresa sueca Pingdom, realizou uma coleta de dados afim de dimensionar o uso da internet no mundo todo. Dentre os os dados destacam-se: (i) Haviam 2,5 bilhões de internautas no mundo; (ii) Haviam 634 milhões de sites ativos; (iii) Haviam 144 bilhões de e-mails trafegando diariamente;  e (iv) 1,3 exabytes era a quantidade de dados trafegados mensalmente nas redes móveis.   

Com o avanço da internet e consequentemente dos computadores, empresas nasceram e desenvolveram poderosos softwares privados tais como OSX(R) da Apple Inc.,Windows(R) da Microsoft. Posteriormente sugiram iniciativa denominas software livre, onde qualquer pessoa têm acesso ao código fonte e dependendo das restrições da licença do software, estes software podem ser modificados e vendidos. Dentre estes softwares os sistemas operacionais baseados em Linux, e temos softwares voltados a internet como o Moodle, objeto de estudo deste trabalho. O Moodle teve sua primeira release em 2001, pelo educador Australiano e cientisca computacional Martin Dougiamas \cite{moodle}. O Moodle é um sistema voltado a criação de comunidades on-line voltadas a aprendizagem colaborativa. Na qual seu desenvolvimento é guiado pela filosofia pedagógica socio-construtiva sustentada em quatro conceitos-chave (MOODLE, 2015): (i) Construtivismo - pessoas constroem colaborativamente novos conhecimentos; (ii) Construcionismo - aprendizado efetivo quando é construído algo que outros podem testar; (iii) Construcionismo Social - é o sentido mais amplo dos conceitos anteriores; e (iv) Ligado e Separado - este baseado nas motivações das pessoas em uma determinada discussão. O capitulo  4 apresenta mais detalhes acerca do Moodle.

\section{Objetivos}
Tendo em vista o contexto e o problema apresentado, este projeto visa prioritariamente o aperfeiçoamento das ontologias existentes na plataforma Moodle, de maneira a contribuir na evolução no tocante a web semântica a plataforma.
\subsection{Objetivo geral}
	Desenhar um ambiente computacional baseado no Moodle visando a construção de uma pesquisa de perfis acadêmicos
\subsection{Objetivos epecíficos}
\begin{itemize}
  \item Contribuir na evolução no ponto de vista Semântico da plataforma Moodle.
  \item Evoluir ontologias existentes para identificar perfis acadêmicos.
	\item Fixar conceitos importantes expostos em sala de aula.
\end{itemize}

\section{Metodologia}
A metodologia adotada neste trabalho consiste em 4 etapas : (i) Levantar referencial teórico; (ii) Levantar estado da Arte; (iii) Definir  escopo; (iv) Elaborar  Mecanismo baseado em ontologias. Como descrito abaixo.

\textbf{Levantar referencial teórico}:
Esta etapa consiste buscar potenciais bibliografias que possam ser utilizadas no desenvolvimento deste trabalho. 

\textbf{Levantar estado da arte}: 
Esta etapa consiste em verificar o estado atual da plataforma Moodle.

\textbf{Definir escopo}: 
Consistem e definir as características do mecanismo baseado em ontologias proposto.

\textbf{Elaborar Mecanismo baseado em ontologia}:
Consiste em desenvolver um Mecanismo,teórico , baseado em ontologias e aplicado à plataforma Moodle.






%O mercado referente aos áudiobooks tem crescido
%
%No capítulo \ref{cap2}, é apresentado os objetivos pretendidos para o qual este trabalho %foi motivado.%
%
%No capítulo \ref{cap3}, é apresentada toda a revisão bibliográfica onde foram revistos %monografias, dissertações, livros, publicações em \textit{websites} e especificações %necessários para o entendimento e desenvolvimento do projeto.%
%
%No capítulo \ref{cap4}, serão apresentados todas as etapas realizadas para a especificação %do formato e para o desenvolvimento do Editor bem como as ferramentas utilizadas com %suporte no processo de desenvolvimento e pesquisa.%
%
%No capítulo \ref{cap5}, serão apresentados os resultados obtidos no projeto, mostrando o %arquivo gerado pelo Editor.%
%
%No capítulo \ref{cap6}, 
%E, por fim, no capítulo \ref{cap7},

%Este documento apresenta considerações gerais e preliminares relacionadas 
%à redação de relatórios de Projeto de Graduação da Faculdade UnB Gama 
%(FGA). São abordados os diferentes aspectos sobre a estrutura do trabalho, 
%uso de programas de auxilio a edição, tiragem de cópias, encadernação, etc.

