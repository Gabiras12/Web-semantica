\chapter[Introdução]{Introdução}\label{cap1}
		
Atualmente, a quantidade de dados - livros, imagens, audios, filmes - disponíveis, na Internet, é muito grande, e vem aumentando rapidamente. Devido a este crescimento, torna-se fundamental a aplicação de técnicas para melhoria no tratamento e organização destes dados, atuando principalmente na sua seleção, processamento, recuperação e disseminação.\cite{UVG}

No meio dos dados encontrados na web, podemos localizar o audiobook, que é a versão em áudio de livros impressos. Também conhecido como “livro falado”, o audiobook tem sido conhecido aos poucos no Brasil e sido utilizado como suporte para que pessoas que não possuem o hábito de ler possam conhecer algumas obras literárias.

Atualmente uma das maiores necessidades no âmbito do audiobook é organização do conteúdo em uma forma similar à organização de livro, uma vez que no audio não há sumário e a busca por uma determinado coteúdo é feita manualmente. Nestes casos, o esforço exercido, a quantidade de tempo e a acurácia na busca dentro de um audiobook trás impacto direto na eficiência desse tipo de material.

Segundo Almeida\cite{UVG}, técnicas de tratamento e organização de dados podem ser classificadas de diversas formas, por exemplo, a partir de seus termos, em glossários ou dicionários, por classificação ou categorias, através de taxonomias, ou a partir de conceitos e seus relacionamentos, utilizando ontologias, tesauros ou redes semânticas.\cite{TSW}

Ontologia é uma técnica de organização de informações que vem recebendo especial atenção nos últimos anos, principalmente no que diz respeito à representação formal de conhecimento. Geralmente criadas por especialistas, tendo sua estrutura baseada na descrição de conceitos e dos relacionamentos semânticos entre eles, as ontologias geram uma especificação formal e explícita de uma conceitualização compartilhada.


\section{Apresentação do Documento do Projeto}

FAZER ESSA SECAO QUANDO O DOCUMENTO TIVER UMA ESTRUTURA COMPLETA

\section{Descrição do Problema}

Atualmente os sistemas de audiobooks representam um grande avanço em número de usuários, isto devido a popularização de dispositivos móveis, que facilitam o acesso a estes meios de informação. Criando assim uma grande demanda de mercado no setor.

Porém estes audiobooks ainda encontram-se em desenvolvimento, sem uma organização estrutural destes dados, necessitando de técnicas que possibilitem uma maior organização, como sumários, glossários e capítulos. Além destas estruturas, necessita-se de maiores informações para, assim, ocasionar uma maior interatividade com o usuário. Dentre estas estruturas, o usuário pode querer ou necessitar utilizar buscas em/por determinados assuntos como, por exemplo, buscas por palavras presentes no áudio, ou buscas por termos de certa relevância no audiobook enriquecendo, desta forma, o uso destas ferramentas.

Assim sendo, tem-se a necessidade de estruturação destas informações em uma ontologia, que possibilite ao usuário encontrar de modo mais eficaz alguns dados de interesse. 

\section{Contexto}

O audiobook é uma forma inovadora de acesso à leitura \cite{audiobooksuporte}. Mesmo para aquelas pessoas que possuem o hábito de leitura, este hábito vem se perdendo devido ao ritmo acelerado das grandes cidades do mundo de hoje. Por falta de tempo, as pessoas tem feito uso de \textit{audiobooks} para estarem ``lendo'' enquanto passam horas dirigindo em extensos engarrafamentos ou enquanto fazem exercícios físicos \cite{audiobookinovacao}. Isso é possível pois a portabilidade do audiobook em comparação aos livros impressos é muito maior, pois o local não parece importar.

Desde 1950, a apreciação pela literatura falada vem crescendo e se tornando tradição nos Estados Unidos. Mas foi em 1980 que realmente os audiobooks ganharam corpo. Segundo \cite{teixeira} havia, em 2006, cerca de 30 mil títulos de audiobooks nos Estados Unidos apenas no site da Audible, companhia da Amazon. Neste ano, a companhia possui mais de 150 mil títulos disponíveis para download \cite{audible}. Este número mostra como a influência dos \textit{audiobooks} vem crescendo no mercado norte-americano.

O crescimento do uso de \textit{audibooks} não tem ocorrido apenas nos Estados Unidos. Na Europa, os livros falados também tem ganhado força, sendo mais populares na Alemanha. Neste país, o hábito virou moda na década de 90. Podemos citar o Instituto Goethe que é um instituição sem fins lucrativos que visa a disseminação do idioma e cultura alemã. Eles possuem \textit{streaming} de diversas obras literárias alemãs disponíveis em seu site. Dois eventos populares na Alemanha, a grande Feira Internacional de Livros de Leipzig e o Festival Internacional de Literatura de Berlim, contam com estandes de venda voltados à apresentação de \textit{audiobooks} \cite{dw}.

Oscar Niemeyer, antes mesmo do uso dos CDs, disse ``Quando quero ler, eu ouço. Pago uma pessoa para gravar os livros em fitas e depois, quando sinto vontade, as coloco para tocar'' \cite{audiobookinovacao}. Aqui no Brasil, o uso de \textit{audiobooks} começou um pouco mais tarde e em um ritmo um pouco mais lento se comparado com Estados Unidos e Europa. No entanto, nos últimos anos os livros falados tem sido cada vez mais utilizados pela população brasileira. O mercado brasileiro tem correspondido a esta nova disseminação cultural \cite{farias}. Isto mostra que a cultura de livros falados no Brasil veio para ficar. Plugme e Audiolivro Editora sao editoras que tem investido em audiolivros no Brasil. Em 2008, quando o mercado ainda era considerado tímido, a Plugme registrou uma venda de setecetnos livros por mês contra mil livros da Audiolivro \cite{audiobooksuporte}. A Audiolivro conta com mais de 900 títulos. Voltado para a literatura infantil, a editora RHJ reúne mais de 200 publicações e premiações recebidas no exterior tais como \textit{White RAvens} na Alemanha, \textit{Octogone} na França, \textit{BIB Plaque} na Eslováquia, \textit{The Noma Concours} no Japão e \textit{The Ibby Honour List Diploma} no Canadá. A editora RHJ também possui premiações no Brasil \cite{rhj}.

A Universidade Falada é um portal de iniciativa privada que visa difundir cultura pelo Brasil com distribuição de conteúdo em áudio viabilizado pela \textit{Editora Alyá} com preços acessíveis e facilidade para a aquisição. Este projeto conta com mais de mil e trezentos audiolivros e estimam mais de cinco mil horas de áudio disponíveis em formato mp3. \cite{universidadefalada}.

\begin{citacao}
O projeto Universidade Falada® pretende democratizar a cultura. Facilitar o acesso, em formato audio e audiolivro, a grandes obras da literatura nacional e internacional à população mais afastada dos grandes centros culturais do país. Nossa missão é ajudar pessoas, oferecer conhecimento e cultura. Discutir temas velhos e novos, ensinar e filosofar. Agregar valor ao ser humano. É isso que nós editores, autores e palestrantes desejamos desta empreitada. Nossos preços permitem que qualquer brasileiro com acesso a internet possa adquirir nossos produtos. Sem exceção \cite{universidadefalada}.
\end{citacao}

Em outubro deste ano, a \textit{PublishNews} publicou uma matéria informando a chegada de duas plataformas de audiolivros para smartphones e tablets no mercado brasileiro. A UBook é uma delas e já está disponível para plataforma iOS e Android. Adotando o serviço de subscrição, os usuários possuem acesso ilimitado a mais de mil obras. Segundo os desenvolvedores, cinco dias após seu lançamento, a plataforma já possuia mais de vinte mil usuários. Uma outra alternativa é a TocaLivros que optou pela venda unitária do audiolivro em formato digital \cite{publishnews}. Esta pretende lançar sua plataforma em algumas horas como mostra a Figura \ref{lanctocalivros}.

 \begin{figure}[ht]
	\centering
		\includegraphics[keepaspectratio=true,scale=0.3]{figuras/lanctocalivros.eps}
	\caption{Captura de tela tirada às 23:24:04 do dia 11/11/14 \cite{tocalivros}}
	\label{lanctocalivros}
\end{figure}

%O mercado referente aos áudiobooks tem crescido
%
%No capítulo \ref{cap2}, é apresentado os objetivos pretendidos para o qual este trabalho %foi motivado.%
%
%No capítulo \ref{cap3}, é apresentada toda a revisão bibliográfica onde foram revistos %monografias, dissertações, livros, publicações em \textit{websites} e especificações %necessários para o entendimento e desenvolvimento do projeto.%
%
%No capítulo \ref{cap4}, serão apresentados todas as etapas realizadas para a especificação %do formato e para o desenvolvimento do Editor bem como as ferramentas utilizadas com %suporte no processo de desenvolvimento e pesquisa.%
%
%No capítulo \ref{cap5}, serão apresentados os resultados obtidos no projeto, mostrando o %arquivo gerado pelo Editor.%
%
%No capítulo \ref{cap6}, 
%E, por fim, no capítulo \ref{cap7},

%Este documento apresenta considerações gerais e preliminares relacionadas 
%à redação de relatórios de Projeto de Graduação da Faculdade UnB Gama 
%(FGA). São abordados os diferentes aspectos sobre a estrutura do trabalho, 
%uso de programas de auxilio a edição, tiragem de cópias, encadernação, etc.

