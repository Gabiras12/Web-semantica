\chapter[metodologia]{Metodologia}\label{cap}
Nossa métodologia será baseada em processos de engenharia de software, definindo quatro fases principais, iniciação, elaboração, desenvolvimento, e terminando pela transição.

\begin{itemize}
 	\item Iniciação: Nesta fase busca-se conhecer o problema em questão, além de um estudo de viabilidade. Basicamente nesta fase os esforços da equipe são destinados a pesquisa sobre o estado atual da ontologia, como “Necessidade da ontologia”, “Quem seriam os beneficiados”, “Existem ontologias que respondem nossas necessidade”. Assim sendo nesta fase busca-se levantar o referencial teórico, objetivos, cenário a ser aplicado, interessados, além de ontologias existentes.
 	\item Elaboração: Nesta etapa fase basicamente a especificação da documentação, aqui seriam definidos o escopo do projeto, bem como a metodologia aplicada ao desenvolvimento da ontologia, além de uma analise técnica dos requisitos necessários a implantação da ontologia.
	\item Desenvolvimento: Nesta fase acontece a implantação em si da ontologia atraves de softwares como o “Protégé”, ferramenta de modelagem de ontologias, deste modo, nesta fase acontece a conceitualização da ontologia em um produto final. Assim sendo segue algumas atividades desta fase como enumerar termos, definir classes, além de outros refinamentos.
	\item Transição: Fase final do projeto, nesta fase acontece o refinamento do produto, modificando e melhorando nos pontos em que for necessário uma intervenção, assim analisando possíveis melhorias no projeto e entregando o produto final.
\end{itemize}